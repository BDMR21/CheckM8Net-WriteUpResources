\documentclass[journal, a4paper]{IEEEtran}
\usepackage[T1]{fontenc}       % Encodage le plus étendu
\usepackage[utf8]{inputenc}    % Source Unicode en UTF-8

\usepackage[cyr]{aeguill}
\usepackage[francais]{babel} % Pour la redaction du document en francais


% some very useful LaTeX packages include:

%\usepackage{cite}      % Written by Donald Arseneau
                        % V1.6 and later of IEEEtran pre-defines the format
                        % of the cite.sty package \cite{} output to follow
                        % that of IEEE. Loading the cite package will
                        % result in citation numbers being automatically
                        % sorted and properly "ranged". i.e.,
                        % [1], [9], [2], [7], [5], [6]
                        % (without using cite.sty)
                        % will become:
                        % [1], [2], [5]--[7], [9] (using cite.sty)
                        % cite.sty's \cite will automatically add leading
                        % space, if needed. Use cite.sty's noadjust option
                        % (cite.sty V3.8 and later) if you want to turn this
                        % off. cite.sty is already installed on most LaTeX
                        % systems. The latest version can be obtained at:
                        % http://www.ctan.org/tex-archive/macros/latex/contrib/supported/cite/

\usepackage{graphicx}   % Written by David Carlisle and Sebastian Rahtz
                        % Required if you want graphics, photos, etc.
                        % graphicx.sty is already installed on most LaTeX
                        % systems. The latest version and documentation can
                        % be obtained at:
                        % http://www.ctan.org/tex-archive/macros/latex/required/graphics/
                        % Another good source of documentation is "Using
                        % Imported Graphics in LaTeX2e" by Keith Reckdahl
                        % which can be found as esplatex.ps and epslatex.pdf
                        % at: http://www.ctan.org/tex-archive/info/

%\usepackage{psfrag}    % Written by Craig Barratt, Michael C. Grant,
                        % and David Carlisle
                        % This package allows you to substitute LaTeX
                        % commands for text in imported EPS graphic files.
                        % In this way, LaTeX symbols can be placed into
                        % graphics that have been generated by other
                        % applications. You must use latex->dvips->ps2pdf
                        % workflow (not direct pdf output from pdflatex) if
                        % you wish to use this capability because it works
                        % via some PostScript tricks. Alternatively, the
                        % graphics could be processed as separate files via
                        % psfrag and dvips, then converted to PDF for
                        % inclusion in the main file which uses pdflatex.
                        % Docs are in "The PSfrag System" by Michael C. Grant
                        % and David Carlisle. There is also some information
                        % about using psfrag in "Using Imported Graphics in
                        % LaTeX2e" by Keith Reckdahl which documents the
                        % graphicx package (see above). The psfrag package
                        % and documentation can be obtained at:
                        % http://www.ctan.org/tex-archive/macros/latex/contrib/supported/psfrag/

%\usepackage{subfigure} % Written by Steven Douglas Cochran
                        % This package makes it easy to put subfigures
                        % in your figures. i.e., "figure 1a and 1b"
                        % Docs are in "Using Imported Graphics in LaTeX2e"
                        % by Keith Reckdahl which also documents the graphicx
                        % package (see above). subfigure.sty is already
                        % installed on most LaTeX systems. The latest version
                        % and documentation can be obtained at:
                        % http://www.ctan.org/tex-archive/macros/latex/contrib/supported/subfigure/

\usepackage{url}        % Written by Donald Arseneau
                        % Provides better support for handling and breaking
                        % URLs. url.sty is already installed on most LaTeX
                        % systems. The latest version can be obtained at:
                        % http://www.ctan.org/tex-archive/macros/latex/contrib/other/misc/
                        % Read the url.sty source comments for usage information.

%\usepackage{stfloats}  % Written by Sigitas Tolusis
                        % Gives LaTeX2e the ability to do double column
                        % floats at the bottom of the page as well as the top.
                        % (e.g., "\begin{figure*}[!b]" is not normally
                        % possible in LaTeX2e). This is an invasive package
                        % which rewrites many portions of the LaTeX2e output
                        % routines. It may not work with other packages that
                        % modify the LaTeX2e output routine and/or with other
                        % versions of LaTeX. The latest version and
                        % documentation can be obtained at:
                        % http://www.ctan.org/tex-archive/macros/latex/contrib/supported/sttools/
                        % Documentation is contained in the stfloats.sty
                        % comments as well as in the presfull.pdf file.
                        % Do not use the stfloats baselinefloat ability as
                        % IEEE does not allow \baselineskip to stretch.
                        % Authors submitting work to the IEEE should note
                        % that IEEE rarely uses double column equations and
                        % that authors should try to avoid such use.
                        % Do not be tempted to use the cuted.sty or
                        % midfloat.sty package (by the same author) as IEEE
                        % does not format its papers in such ways.

\usepackage{amsmath}    % From the American Mathematical Society
                        % A popular package that provides many helpful commands
                        % for dealing with mathematics. Note that the AMSmath
                        % package sets \interdisplaylinepenalty to 10000 thus
                        % preventing page breaks from occurring within multiline
                        % equations. Use:
%\interdisplaylinepenalty=2500
                        % after loading amsmath to restore such page breaks
                        % as IEEEtran.cls normally does. amsmath.sty is already
                        % installed on most LaTeX systems. The latest version
                        % and documentation can be obtained at:
                        % http://www.ctan.org/tex-archive/macros/latex/required/amslatex/math/

\usepackage{lipsum}
\usepackage{hyperref}
\usepackage[backend=biber,style=science]{biblatex}
\addbibresource{bibliography.bib}
\usepackage{amssymb} % Dummy text


% Other popular packages for formatting tables and equations include:

%\usepackage{array}
% Frank Mittelbach's and David Carlisle's array.sty which improves the
% LaTeX2e array and tabular environments to provide better appearances and
% additional user controls. array.sty is already installed on most systems.
% The latest version and documentation can be obtained at:
% http://www.ctan.org/tex-archive/macros/latex/required/tools/

% V1.6 of IEEEtran contains the IEEEeqnarray family of commands that can
% be used to generate multiline equations as well as matrices, tables, etc.

% Also of notable interest:
% Scott Pakin's eqparbox package for creating (automatically sized) equal
% width boxes. Available:
% http://www.ctan.org/tex-archive/macros/latex/contrib/supported/eqparbox/

% *** Do not adjust lengths that control margins, column widths, etc. ***
% *** Do not use packages that alter fonts (such as pslatex).         ***
% There should be no need to do such things with IEEEtran.cls V1.6 and later.


% En-tête et pied de page
%\usepackage{lastpage}
%\usepackage{fancyhdr}
%\pagestyle{fancy}
%\renewcommand{\sectionmark}[1]{\markright{#1}}
%\fancyhead{}
%%\fancyhead[RO,LE]{\slshape\footnotesize\nouppercase{\rightmark}}
%\fancyhead[LO,RE]{\thetitle}
%\fancyfoot{}
%%\fancyfoot[LO,RE]{\footnotesize\texttt{\thefilename}\\ \textit{\now}}
%%\fancyfoot[C]{-~\thepage~/~\pageref{LastPage}~-}
%\fancyfoot[RO,LE]{\raisebox{-2mm}{\includegraphics{structure/barrette-original}}}
%%
%\fancypagestyle{plain}{ %  Première page ----------------------
%  \fancyhead{}
%  \renewcommand{\headrulewidth}{0pt}
%  \fancyheadoffset[R]{15mm}
%  \fancyhead[L]{
%    \raisebox{-7mm}{
%      \parbox{\textwidth}{
%        \includegraphics{structure/barrette-original} \\ \\
%        \fontsize{8pt}{10pt}\selectfont
%        \sffamily\color{Pantone287}
%        FACULTÉ DES SCIENCES       \\
%        DÉPARTEMENT D'INFORMATIQUE   
%      }
%    }
%  }
%  \fancyhead[R]{
%    \raisebox{-10mm}[0pt][0pt]{\includegraphics[width=120mm]{structure/ULB-ligne-gauche}}
%  }
%  \fancyfoot{}
%  %\fancyfoot[L]{\raisebox{0mm}{}\color{Pantone287}\footnotesize\texttt{\thefilename}\\ \textit{\now}}
%  %\fancyfoot[C]{-~\thepage~/~\pageref{LastPage}~-}
%  \fancyfoot[R]{
%    \raisebox{-12pt}{\includegraphics[height=\footskip]{structure/sceau-mini-b-quadri}}
%  }
%} % Fin de première page
% ---------------------------------------------------------------------------


% Your document starts here!
\begin{document}

% Define document title and author
	\title{Rapport Projet BlockChain}
	\author{Ben David MALYANE - 471086}
%	\thanks{Superviseur: xxx}}
	\markboth{INFO-F308}{}
	\maketitle

% Write abstract here
\begin{abstract}
	Ce rapport présente le travail réalisé dans le cadre du projet d'informatique 3 trandisciplinaire qui consistait
	à implémenter une application de type BlockChain à usage ludique.
	Cette application devrait permettre de décentraliser la gestion des résultats de partie de jeu quelconque.
	Nous parlerons des choix de conception et d'implémentation qui ont été faits pour réaliser cette application.
\end{abstract}

% Each section begins with a \section{title} command
\section{Introduction}\label{sec:introduction}
	% \IEEEPARstart{}{} creates a tall first letter for this first paragraph
	\IEEEPARstart{C}{e} rapport présente le travail que nous avons réalisé dans le cadre du projet d'informatique 3 trandisciplinaire visant à réaliser une blockchain à usage ludique.
	L'objectif principal était de décentraliser la gestion des résultats de jeu quelconque, en  utilisant  les principes fondamentaux de la BlockChain pour assurer l'équité,
	la transparence et la sécurité.

	Au cours de ce rapport :
	%itemize with bullets points
	\begin{itemize}
		\item[•] Nous présenterons les choix de conception et d'implémentation que nous avons fait afin de réaliser cette application,
		\item[•] Nous explorerons les différentes étapes du processus de développement en mettant l'accent sur les technologies utilisées
	et les architectures mises en place,
		\item[•] Nous aborderons les problèmes rencontrés ainsi que les solutions qui ont été apportées,
		\item[•] Nous présenterons les fonctionnalités clés de notre application, telles que la gestion transparente
		des participant, la vérification des résultats et la gestion des récompenses base de nos incitatifs ainsi que la sécurité des données,
		\item[•] Nous expliquerons les mécanismes de consensus et les protocoles de sécurité mis en place pour assurer la fiabilité de notre application.
	\end{itemize}


\section{BlockChain}\label{sec:blockchain}

	\subsection{\textbf{Définition}}\label{subsec:definition}

		La Blockchain est une technologie de stockage et de transmission d'informations, transparente, sécurisée,
		et fonctionnant sans organe central de contrôle.
	Elle agit comme un grand livre de compte, ouvert et consultable par tous les utilisateurs, qui contient l'historique
		de toutes les transactions effectuées entre ses utilisateurs depuis sa création~\cite{blockchain}.
	Utilisée par les crypto-monnaies telles que le Bitcoin, la Blockchain garantit l'intégrité et la sécurité des données
	en permettant un stockage décentralisé et en rendant presque impossible les modifications rétroactives.
	Cette technologie offre ainsi un moyen fiable et transparent de gérer et de transmettre des informations,
	ouvrant la voie à de nombreuses applications dans divers domaines.

	\subsection{\textbf{Structure d'une BlockChain: Composants et fonctionnement}}\label{subsec:structure}

			Une BlockChain, comme son nom l'indique, est une série de blocs de données chainés les uns aux autres par des liens de hachage cryptographiques.
		Elle est composée de plusieurs éléments clés qui travaillent ensemble pour assurer la sécurité et la fiabilité du système.
		Les principaux éléments de la structure d'une Blockchain sont les suivants:
			\begin{enumerate}
				\item \textbf{Blocs:} Chaque bloc contient un ensemble de transactions, un hachage de bloc précédent et un hachage de bloc courant.
				\item \textbf{Chaîne de blocs:} La chaîne de blocs est une liste de blocs liés les uns aux autres par des hachages cryptographiques.
				\item \textbf{Transactions:} Les transactions représentent les opérations effectuées par les utilisateurs de la BlockChain.
				Cela peut inclure des transferts de crypto-monnaies, des contrats intelligents ou d'autres types d'actions enregistrées dans la BlockChain.
				\item \textbf{Mécanisme de consensus:} Le mécanisme de consensus est un protocole qui permet aux nœuds de la BlockChain de s'accorder sur l'état de la BlockChain.
				\item \textbf{Cryptographie:} La cryptographie est utilisée pour sécuriser les données et les transactions dans la BlockChain.
				\item \textbf{Réseau P2P:} Le réseau P2P est un réseau décentralisé qui permet aux nœuds de la BlockChain de communiquer entre eux.
				\item \textbf{Contrats intelligents:} Les contrats intelligents sont des programmes autonomes qui s'exécutent sur la BlockChain.
			\end{enumerate}

	\subsection{\textbf{Le concept des signatures électroniques: Assurer l'authenticité et l'intégrité des données électroniques}}\label{subsec:signature}
			Les signatures électroniques sont un concept clé de la cryptographie moderne.
		Elles sont utilisées pour assurer l'authenticité, l'intégrité et la non-répudiation des données électroniques.
		Elles reposent sur le principe de la cryptographie asymétrique, qui utilise une paire de clés (publique et privé) pour chiffrer et déchiffrer les données,
		permettant ainsi de vérifier l'identité de l'auteur d'un document ou d'une transaction numérique et d'assurer l'intégrité du contenu depuis sa signature.

		Le proecessus de signature électronique se déroule en trois étapes:
			\begin{enumerate}
				\item \textbf{Génération de la paire de clés:} L'auteur génère une paire de clés (publique et privée) et rend publique sa clé publique.
				\item \textbf{Signature:} Pour créer une signature unique, l'auteur utilise sa clé privée pour chiffrer l'empreinte numérique du document
				à l'aide d'un algorithme de hachage cryptographique.
				\item \textbf{Vérification:} Le destinataire vérifie la signature avec la clé publique de l'auteur.
			\end{enumerate}

		La comparaison de l'empreinte numérique déchiffrée avec celle du document original permet de vérifier l'authenticité et l'intégrité du document.
		Cette signature empêche également l'auteur de nier avoir signé le document, car la clé privée est unique et ne peut être utilisée que par son propriétaire.

% Main Part
\section{CheckM8Net : Blockchain ludique}\label{sec:checkm8net-:-blockchain-ludique}
	Cette section présente les aspects essentiels du projet, tels que les choix de conception, les problèmes rencontrés et les solutions apportées.
	Nous commencerons par décrire en détail la structure de notre blockchain et la gestion du réseau.
	Ensuite, nous évaluerons les performances de notre implémentation et nous discuterons des limites, ainsi que des hypothèses et
	des simplifications que nous avons faites lors de la conception, telles que les blocs, la présence d'un arbitre et les droits accordés aux joueurs.
	Enfin nous ferons une analyse approfondie des incitatifs mis en place sera également présentée, en examinant leur efficacité pour résoudre
	les problèmes et en envisageant les éventuels contournements par un joueur malveillant.

	\subsection{Choix de conception}\label{subsec:choix-de-conception}
		\begin{center}
			\subsubsection*{\textbf{Structure de notre blockchain}}\label{subsubsec:structure-de-notre-blockchain}
		\end{center}

			Notre blockchain implémente une structure plutôt habituelle, avec des blocs, des transactions,
		des signatures électroniques ainsi qu'un mécanisme de consensus, à la seule différence que nous simplifions la structure
		interne des blocs en ne considérant qu'une seule transaction par bloc.

		Nous manipulons différents types de transactions, mais ici nous mettrons l'accent sur les deux principales qui sont les suivantes:
			\begin{enumerate}
				\item \textbf{Transaction de nouvelles parties:} Cette transaction est utilisée par les joueurs ou les arbitres  pour créer une nouvelle partie de jeu.
					Elle se compose des informations suivantes:
					\begin{itemize}
						\item[•] \textbf{signature:} Signature de la transaction.
						\item[•] \textbf{timestamp:} Horodatage de la transaction.
						\item[•] \textbf{sender:} Adresse du joueur ou de l'arbitre qui a initié la transaction.
						\item[•] \textbf{game id:} Identifiant de la partie.
						\item[•] \textbf{player 1:} Clé publique du joueur 1.
						\item[•] \textbf{player 2:} Clé publique du joueur 2.
						\item[•] \textbf{referees:} Liste des clés publiques des arbitres.
					\end{itemize}

				\item \textbf{Transaction de résulats de partie:} Cette transaction est utilisée par les arbitres pour déclarer le résultat d'une partie.
					Elle se compose des informations suivantes:
				    \begin{itemize}
						\item[•] \textbf{signature:} Signature de la transaction.
						\item[•] \textbf{timestamp:} Horodatage de la transaction.
						\item[•] \textbf{sender:} Adresse du joueur ou de l'arbitre qui a initié la transaction.
						\item[•] \textbf{game id:} Identifiant de la partie.
						\item[•] \textbf{player 1:} Clé publique du joueur 1.
						\item[•] \textbf{player 2:} Clé publique du joueur 2.
						\item[•] \textbf{result:} Résultat de la partie.
						\item[•] \textbf{referees:} Liste des clés publiques des arbitres.
					\end{itemize}
			\end{enumerate}

			Nous avons défini un mécanisme de consensus qui permet de valider la version de la blockchain la plus pertinente
			c'est-à-dire la plus longue, soit celle avec le score le plus élevé.
			Le score d'un bloc est calculé comme suit:
			\begin{equation}
				\label{eq:equation}
				\text{Score bloackchain} = #\text{transactions|bloc}
			\end{equation}

			Ainsi si le résultat de la partie est validé par tous les participants de la partie, alors le bloc est validé et ajouté à la blockchain.
			Sinon, si il y a un désaccord, le bloc est tout de même accepté mais les joueurs et les arbitres sont pénalisés.
			Ce mécanisme nous permet alors d'établir les scores ELO pour les joueurs et les trusts pour les arbitres.

		\begin{center}
			\subsubsection*{\textbf{Gestion du réseau}}\label{subsubsec:gestion-du-réseau}
		\end{center}

				En ce qui concerne l'architecture réseau de notre blockchain, nous avons opté pour une approche peer-to-peer (P2P)
			pour la simple raison que cela nous permet d'avoir le réseau décentralisé que nous recherchions.
			Cette architecture décentralisée présente de nombreux avantages, notamment une plus grande résilience face aux attaques et surtout
			une distribution équitables des données dans le réseau.
			Ainsi les nodes servers peuvent se synchroniser entre eux.

				Pour ce qui est des nodes servers, nous avons choisi d'utiliser le framework \textbf{Django} ~\cite{Django} sur Python afin de développer notre blockchain.
			Au moyen de Django, nous avons créé une API REST, qui permet aux nodes du réseau de communiquer entre eux de mannière asynchrone.
			Cette communication entre les nodes est basée sur un mécanisme de gossip, assurée par les signaux de Django.
			Lorsqu'un node server reçoit un nouveau bloc, il le propage aux nodes dont il a connaissance, qui eux-mêmes le propagent à leur tour.
			Cela nous permet de toujours avoir la même version de la blockchain sur tout le réseau après validation d'un bloc.

				Il nous a fallu également intégrer un système d'ordonnancement des messages en utilisant le système de
			messagerie RabbitMQ ~\cite{RabbitMQ}, de sorte à pouvoir maintenir l'intégrité des données et d'éviter les pertes ou
			les incohérences lors de la transmission des messages.
			Associé à RabbitMQ, nous avons utilisé le système de gestion de tâches Celery ~\cite{Celery}, qui nous a permis de gérer
			les tâches asynchrones et de pouvoir les exécuter en arrière-plan.

				Nos avons choisis d'héberger nos serveurs sur des machines virtuelles fournies par \textbf{DigitalOcean}.
			Et pour faciliter l'accès à nos serveurs (nodes servers), nous avons enregistré un nom de domaine \textbf{\url{http://www.checkm8.cloud/}}.
			Ce domaine nous permet de répertorier dans un A record, les adresses IP de nos nodes servers et par la méthode de round-robin,
			un serveur est choisi aléatoirement pour répondre à une requête utilisateur.

	\subsection{Problèmes rencontrés et solutions apportées}\label{subsec:problèmes-rencontrés-et-solutions-apportées}
		\begin{center}
			\subsubsection*{\textbf{Problèmes rencontrés}}\label{subsubsec:problèmes-rencontrés}
		\end{center}

		\begin{itemize}
			\item[•] Structure d'un bloc
			\item[•]
		\end{itemize}

		\begin{center}
			\subsubsection*{\textbf{Solutions apportées}}\label{subsubsec:solutions-apportées}
		\end{center}

	\subsection{Performances}\label{subsec:performances}
		\begin{center}
			\subsubsection*{\textbf{Performances de notre implémentation}}\label{subsubsec:performances-de-notre-implémentation}
		\end{center}

		\begin{center}
			\subsubsection*{\textbf{Limites, hypothèses et simplifications}}\label{subsubsec:limites,-hypothèses-et-simplifications}
		\end{center}
%			Il est important de noter que, bien que nous ayons prévu la possibilité d'intégrer la fonctionnalité de vote
%			pour les joueurs dans l'application mobile, nous ne l'avons pas encore implémentée.
%			Cependant, notre API est conçue pour prendre en charge cette fonctionnalité de vote. TODO

	\subsection{Analyse des incitatifs}\label{subsec:analyse-des-incitatifs}
		\begin{center}
			\subsubsection*{\textbf{Analyse des incitatifs mis en place}}\label{subsubsec:analyse-des-incitatifs-mis-en-place}
		\end{center}
			Nous avons pensé CheckM8Net sur base de deux types d'acteurs que sont les joueurs et les arbitres.
			Ce qui nous a mené à avoir différents incitatifs pour chacun de nos acteurs dans le but de les encourager
			à participer de manière active dans le réseau.

			Pour les joueurs, notre blockchain les incite à améliorer leur score ELO en jouant des parties de jeu.
			Le score ELO~\cite{ELO} est un système de classement utilisé dans les jeux compétitifs tels que les échecs, le go, etc.
			Initialement ce score est initialisé à 1000 et sur base d'un algorithme de calcul, plus un joueur gagne des parties,
			son score augmente, et plus il en perd, son score diminue.
			Les joueurs sont donc incités à joueur des parties et surtout à les emporter dans le but d'être le mieux classé.

			Concernant les arbitres, notre blockchain les incite à améliorer leur score d'arbitrage (Trusts) en contribuant
			à des parties de jeu sur notre blockchain.
			Initialement ce trust est initialisé à 0.5 et sur base d'une fonction logistique, on le calcule.
			Cette fonction est basé sur l'état du consensus parmi tous les participants qui ont approuvé ou non
			le résultat de la partie de jeux.
			Ainsi, si un arbitre donne des résultats incorrects, il perd en confiance dans le réseau, sinon il en gagne.

		\begin{center}
			\subsubsection*{\textbf{Contournement par un joueur/arbitre malveillant}}\label{subsubsec:contournement-par-un-joueur-malveillant}
		\end{center}
			Bien que notre système d'incitatifs vise à encourager une participation honnête et à dissuader les comportements malveillants,
			nous sommes potentiellement sujet à certains contournements, telles que:
			\begin{enumerate}
				\item \textbf{Les attaques Sybil:} Ces attaques consistent à créer une grande quantités d'identités ou comptes
					afin d'influencer les résultats dans le but de manipuler les classements~\cite{Sybil}.
				\item \textbf{La manipulation des résultats:} Cette attaque pourrait se produire par la collusion avec d'autres joueurs ou arbitres.
					Un joueur malveillant pourrait tenter de manipuler les résultats afin d'obtenir un meilleure score ELO
					ou un arbitre malveillant pourrait le faire pour obtenir un meilleur trust.
			\end{enumerate}


\section{Conclusion}\label{sec:conclusion}
	Cette section contient un rappel des contributions / de résultats importants de votre article et éventuellement une indication sur les perspectives de recherche future dans le même domaine.

% References
\newpage
\section*{\textbf{Références}}\label{sec:références}
	\printbibliography[heading=none]

\newpage
		
\appendices


% Your document ends here!
\end{document}

les transactions de résultat de partie et ainsi
	de valider le bloc (transaction).